
    \begin{abstract_online}{Embodied locomotion through self-organized frequency locking}{%
        \underline{C. Gros}$^{1}$, F. Kubandt$^{1}$, B. S\'andor$^{1,2}$}{%
        }{%
        $^1$ Institute for Theoretical Physics, Goethe University Frankfurt, Frankfurt a. M., Germany\newline{}$^2$ Department of Physics, Babes-Bolyai University, Cluj-Napoca, Romania}
    Embodied locomotion corresponds to attractors in the sensorimotor loop, that is in the feedback  loop made up by the controlling neurons, the actuators and the reaction of the environment. We  present several types of simulated and real-world robots for which every actuator is controlled  separately by a maximum of two rate encoding neurons. Complex self-organized behavior in the form  of either limit cycles and chaotic attractors arises, leading respectively to regular and exploratory locomotion.  \par  For the case of wheeled robots we find that the individual wheels correspond to regular oscillators  with natural frequencies that are determined by the feedback parameters of the sensorimotor loop.  The individual wheels are in our approach fully independent, using only the own rotational angle as  incoming sensory information. Explicit cross-wheel controlling or coordination is absent. The wheels  are however coupled indirectly via the physics of the body of the robot whenever they experience  non-equal environmental feedbacks.  \par  We find that the indirect coupling of the limit cycles driving the individual wheels leads to cross-wheel frequency locking and, as a consequence, to robust locomotion. We propose self-organized in-and out-of-phase frequency locking of limb movements as a possible mechanism for the generation  of non-trivial gaits.  \par  \begin{center}  \includegraphics[width=\linewidth]{abstracts/txt/figures/gros}  \end{center}  \vspace{-0.5cm}  \noindent  {\small{\bf Train of passively coupled two-wheel cars.} The wheels  $n=1,..,10$ are controlled independently by a single rate encoding neuron that receives as an input $\cos(\varphi_n)$, where  $\varphi_n$ is the rotational angle of the respective wheel. The output of the neuron corresponds to the driving torque. There is no cross-wheel  or cross-car controlling. The parameters of the individual sensorimotor feedback loops, which differ slightly, give rise  to natural frequencies $\nu_n$.  ({\sl left}) The power spectrum of the ten wheels for a train  suspended in the air. The wheels rotate independently with  their respective natural frequencies.  ({\sl right}) The ten wheels rotate on average with the same frequency when the train is moving on the ground, with a residual chaotic component giving rise to higher frequency side bands (beyond  the scale).  (\href{http://itp.uni-frankfurt.de/~gros/Movies/wheelRobot/lpz_wheeled_train.mp4}        {online, click for movie).  }  }  
    
    \end{abstract_online}
    