
        \begin{abstract}{Embodied locomotion through self-organized frequency locking}{%
            C. Gros}{%
            Goethe University Frankfurt, Frankfurt, Germany}{%
            }
        Embodied locomotion corresponds to attractors in the  sensorimotor loop, that is in the feedback loop made up  by the controlling neurons, the actuators and the  reaction of the environment. We present several types of  simulated and real-world robots for which every actuator  is controlled separately by a maximum of two rate encoding  neurons. Complex self-organized behavior in the form of either  limit cycles and chaotic attractors arises, leading  respectively to regular and exploratory locomotion.  \par  For the case of wheeled robots we find that the  individual wheels correspond to regular oscillators  with natural frequencies that are determined by the  feedback parameters of the sensorimotor loop. The  individual wheels are in our approach fully independent,  using only the own rotational angle as incoming sensory  information. Explicit cross-wheel controlling or  coordination is absent. The wheels are however coupled  indirectly via the physics of the body of the robot  whenever they experience non-equal environmental feedbacks.  \par  We find that the indirect coupling of the limit cycles  driving the individual wheels leads to cross-wheel  frequency locking and, as a consequence, to robust locomotion.  We propose self-organized in- and out-of-phase  frequency locking of limb movements as a possible  mechanism for the generation of non-trivial gaits.  
        \end{abstract}
        